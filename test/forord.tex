\chapter*{Forord}
Denne bog er tiltænkt nybegynder (n00bz) inden for *NIX\footnote{*NIX er UNIX eller Linux baseret systemer} verden. 
Som godt kunne tænke sig at få en introduktion i hvad der ligger under den poleret brugergrænseflade. 

\section*{Om forfatteren}
Lars er født i 1989 så et lævn fra det forige årtusind. Lars er uddannet datamatiker fra Aarhus Erhvervs Akademi sommeren 2013. 
Men han start med at læse datalogi ved Aarhus Universitet i 2009, men det var nu lidt for teoretisk, så han skiftet til den mere praktisk 
tilgang og læser nu software ved Aalborg Universtet under School of information and computer technology. Desuden har Lars arbejdet ved flere 
virksomheder der udarbejder webbaseret løsninger og nogle få virksomheder der arbjeder primært med back-end løsnigner. Kort sagt Lars er nørd, 
og bruger det mest af sin tid med softwareudvikling og primært ligger hans interesse inden for back-end, system udvikling og systemhåndtering. 

\section*{Kontakt:}
Hvis du har kommentar, forslag eller andre ting til denne bog, ville det være at fortrække at smide en kommentar på bogens github side, 
som findes her \href{https://github.com/looopTools/intro\_til\_terminal}{github.com/looopTools/intro\_til\_terminal}. Hvor der findes en issue side, 
hvor sådanne ting bliver håndteret. Har du derimod et sprøgsmål til Lars selv kan du kontakt ham på følgende mail adresse: \href{mailto:lnc13.lars@gmail.com}{lnc13.lars@gmail.com}

\section*{Tak til:}
Tak til \href{https://www.linux.dk}{www.linux.dk} og holdet bag, for at give mig 
lov til at bruge siden til at start min serie af guides, som er blevet samlet til den her bog. 
\par Tak til Peter Lyberth og Kim Rostgaard Christensen, som begge står bag linux.dk og som af og til har læst nogle af online guidesne igennem, for stavefejl.
\par Tak til folkne bag \LaTeX, som bogen er opbygget i. 

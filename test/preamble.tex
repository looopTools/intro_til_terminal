\documentclass[a4paper,11pt,fleqn,twoside,openany]{memoir} % Brug openright hvis chapters skal starte på højresider; openany, oneside
%\documentclass[a4paper,11pt]{report} 
\usepackage[danish]{babel}
\usepackage[utf8]{inputenc}
\usepackage[T1]{fontenc}
\usepackage{amsmath,amsfonts,amssymb}
\usepackage[final]{pdfpages}
\usepackage{setspace}
\usepackage{graphicx}
\usepackage{float}
\usepackage[footnote,draft,silent,nomargin]{fixme}
\usepackage{tabularx} % Table

\usepackage{url}
\usepackage{cite}
\usepackage{listings}
\usepackage{color}
\usepackage[margin=3.0cm]{geometry}
\usepackage{lastpage}
\usepackage{fancyhdr} 

\usepackage[hidelinks, breaklinks]{hyperref} % Hyper link
\PassOptionsToPackage{hyphens}{url}\usepackage{hyperref}
\usepackage{titlesec}
\usepackage{afterpage}

% ¤¤ Navngivning ¤¤ %
\addto\captionsdanish{
	\renewcommand\appendixname{Bilag}
	\renewcommand\contentsname{Indholdsfortegnelse}	
	\renewcommand\appendixpagename{Bilag}
	%\renewcommand\cftchaptername{\chaptername~}				% Skriver "Kapitel" foran kapitlerne i indholdsfortegnelsen
	\renewcommand\cftappendixname{\appendixname~}			% Skriver "Bilag" foran bilagene i indholdsfortegnelsen
	\renewcommand\appendixtocname{Bilag}
}

%Paragraph settings
%Korregerer Paragraph setting fra Memoir klassen. 
\setlength{\parskip}{\baselineskip}
\setlength{\parindent}{0pt}

\pagestyle{plain}

% ¤¤ Fjerner den vertikale afstand mellem listeopstillinger og punktopstillinger ¤¤ %
\let\olditemize=\itemize							
\def\itemize{\olditemize\setlength{\itemsep}{-1ex}}
\let\oldenumerate=\enumerate						
\def\enumerate{\oldenumerate\setlength{\itemsep}{-1ex}}

\setcounter{tocdepth}{3}
\setcounter{secnumdepth}{4}

\newcommand\blankpage{%
    \null
    \thispagestyle{empty}%
    \addtocounter{page}{-1}%
    \newpage}

% ¤¤ Kapiteludssende ¤¤ %
\definecolor{numbercolor}{gray}{0.7}			% Definerer en farve til brug til kapiteludseende
\newif\ifchapternonum

\makechapterstyle{jenor}{									% Definerer kapiteludseende -->
  \renewcommand\printchaptername{}
  \renewcommand\printchapternum{}
  \renewcommand\printchapternonum{\chapternonumtrue}
  \renewcommand\chaptitlefont{\fontfamily{pbk}\fontseries{db}\fontshape{n}\fontsize{25}{35}\selectfont\raggedleft}
  \renewcommand\chapnumfont{\fontfamily{pbk}\fontseries{m}\fontshape{n}\fontsize{1in}{0in}\selectfont\color{numbercolor}}
  \renewcommand\printchaptertitle[1]{%
    \noindent
    \ifchapternonum
    \begin{tabularx}{\textwidth}{X}
    {\let\\\newline\chaptitlefont ##1\par} 
    \end{tabularx}
    \par\vskip-2.5mm\hrule
    \else
    \begin{tabularx}{\textwidth}{Xl}
    {\parbox[b]{\linewidth}{\chaptitlefont ##1}} & \raisebox{-15pt}{\chapnumfont \thechapter}
    \end{tabularx}
    \par\vskip2mm\hrule
    \fi
  }
}																						% <--

\chapterstyle{jenor}												% Valg af kapiteludseende - dette kan udskiftes efter ønske


% ¤¤ Indholdsfortegnelse ¤¤ %
\setsecnumdepth{subsubsection}		 					% Dybden af nummerede overkrifter (part/chapter/section/subsection)
\maxsecnumdepth{subsubsection}							% Ændring af dokumentklassens grænse for nummereringsdybde
\settocdepth{subsubsection} 								% Dybden af indholdsfortegnelsen

\definecolor{dkgreen}{rgb}{0,0.6,0}
\definecolor{gray}{rgb}{0.5,0.5,0.5}
\definecolor{mauve}{rgb}{0.58,0,0.82}

\lstset{frame=tb,
  language=sh,%bash,
  aboveskip=3mm,
  belowskip=3mm,
  showstringspaces=false,
  columns=flexible,
  numbers=none,
  numberstyle=\tiny\color{gray},
  breaklines=true,
  breakatwhitespace=true
  tabsize=2
}

\title{Introduktion til Terminal}
\author{Lars Nielsen \\ \href{mailto:lnc13.lars@gmail.com}{lnc13.lars@gmail.com}}
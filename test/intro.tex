\chapter{Introduktion}
Siden den personlige computeres oprindelse, har der været to primær input enheder til computer den ene er pegeværktøjet (musen, pointwheel trackball) og tastaturet. De flest bruger i dag en kombination af de to, til at navigere og bruge deres computere. Så er der folk som os (nørder) som fortrækker at flyttet hænderne mindst muligt fra tastaturet, medmindre vi rækkker hånden ud af kaffen. Vi elsker at kunne bruge vores computere og systemer uden at skulle klikke rundt. Vi har et hemmeligt våben til dette, dette våben er bedere kendt som Terminal eller komandolinje.
\par Dette våben er et lævn fra den gang computer ikke havde en mus og kun tastaturet som input enhed.

Terminal eller Terminalen er et kommandolinje værktøj som findes i forskellig udgaver, men de flest systemer bruger de samme kommandoer. De kommandoer som bliver brugt i denne bog kan bruges på Mac OS X, Linux og BSD, hvis en funktion ikke findes på alle systemer bliver dette naturligvis noteret.

\section{Hvad bliver ikke dækket}
Der er elementer af den kommando linje baseret verden, som ikke bliver dækket i denne bog og med god grund.
\begin{description}
  \item[Server opsætning] er et helt emne for sig selv, og er simpelt hen for stort til at blive dækket i en begynder bog på dette niveau og det varier meget fra system til system.
  \item[Tekstværktøjer] som en del af det at bruge kommandolinjen, 
    er tekstværktøjer og der findes forskellig slags (Vi(m), Emacs, Nano, 
    Pico, ...). Og da der er en kæmpe relegionskrig i mellem specielt Vi(m) 
    og Emacs har jeg valgt ikke at dække dem. I bogen bruges Vi eller Emacs. Jeg vil 
    dog prøve at holde mig til Vi, da flest systemer komme med denne som standard.
  \item[Hacking] er et mere avanceret emne og kræver lidt evner og bliver derfor ikke dækket i denne bog.
  \item[User management] eller brugerhåndtering, er i bund og grund ikke så svært, men variger for meget på tværs af systemer til at give 
    et godt overordnet indblik.
\end{description}

\section{Kommando opbygning}
For at kunne benyttet kommandolinjen eller terminalen, benyttes kommandoer. disse kommandoer har en opbygning og for at benytte disse kommadoer er det ret vigitigt at forstå denne opbygning. En Kommando består pirmært af tre dele, disse tre dele kaldes kommandoen, flag og argumenter.  
\subsection*{Kommandoen}
Er navnet på den kommando som man ønsker at benyttet sig af. navnet er altid den først del af et kommandokald. Eksempler på 
kommandonavne er; \textit{cd, ls, pwd og mv}.
\subsection*{Flag}
Flag er parameter, som benyttes til at ændre eller manipulere det output en kommando retuner disse kommer oftes efter kommandonavnet eksempeltvis; \textit{ls -a} som bliver beskrevet senere. Flag starter oftes med et dash (-) og derefter et enkelt bogstav. En kommando kan sagtens tage flere flag eller ingen, det ser vi eksempler på senere.
\subsection*{Argumenter}
Disse kommer oftes efter flag og er det "element" som en kommando bliver udført, forskellig kommando kan tage mere en et argument. Eksempl på kommando uden flag, men med et argument \textit{ls /test\_2}
\section{MAN - Din ven i mørket}
Når man vil kende sin funktion lidt mere i dybten, bruger man en kommando som hedder \textit{man}. \textit{man} står for manual og giver et overblik over kommando, mulige flag og argumenter. se eksempel neden for
\begin{lstlisting}
	h271:intro_til_terminal tools$ man ls
	LS(1)                     BSD General Commands Manual                    LS(1)

	NAME
	     ls -- list directory contents

	SYNOPSIS
	     ls [-ABCFGHLOPRSTUW@abcdefghiklmnopqrstuwx1] [file ...]

	DESCRIPTION
	     For each operand that names a file of a type other than directory, ls
	     displays its name as well as any requested, associated information.  For
	     each operand that names a file of type directory, ls displays the names
	     of files contained within that directory, as well as any requested, asso-
	     ciated information.

	     If no operands are given, the contents of the current directory are dis-
	     played.  If more than one operand is given, non-directory operands are
	     displayed first; directory and non-directory operands are sorted sepa-
	     rately and in lexicographical order.

	     The following options are available:
\end{lstlisting}
\textit{man} siden kan navgieres med med piletasterne og man kommer udaf den ved at indtaste \textit{q}. Hvis du vil vide lidt mere om \textit{man}, så har \textit{man} selv en man side og den tilgåes ved at skrive \textit{man man}

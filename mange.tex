\chapter{Små gode ting}
I dette kapitel dækkes små "hacks" til at kunne gøre arbejdet med kommandolinjen letter.
\section{Mange kommandoer på engang}
Når man med tiden opdager at flere kommando tit bliver ud ført sammen og i hvis række følge, bliver man træt af at skrive en kommando trykke enter og så skrive næste kommando, og det er her man bliver glad for \textit{\&\&} som tillader en at link kommandoer.
Eksempelvis, hvis man vil lave en mappe også gå direkte ind i den, altså kombinere \textit{mkdir} og \textit{cd}, og i dette eksempels tilfælde også \textit{pwd} kan dette gøres så ledes .
\begin{lstlisting}
	h271:intro_til_terminal tools$ pwd && mkdir kombi_test && cd kombi_test && pwd
	  /Users/tools/Documents/intro_til_terminal
	  /Users/tools/Documents/intro_til_terminal/kombi_test 
\end{lstlisting}
Som udføre alle kommandoerne i række følge. 
\section{ALIAS}
Kommandoer kan være lange og svære og huske, derfor kan man benytte benytte sig af \textit{alias}.
Alias er i sig selv ikke kommando men en funktionalitet, som er en del af ens kommandolinjeværktøj. 
For at benyttet sig af alias, skal der editeres i en fil som kaldes \textit{.bashrc}, denne fil 
kan editeres ved at benytte \textit{VI}. Som eksempel, vil vi benytte kommandoen cd som eksempel, 
også vil vi gerne ændre nuværende mappe til skrivebordet, vi gør så ledes:
\begin{lstlisting}
	h271:intro_til_terminal tools$ vi  ~/.bashrc
\end{lstlisting}
Når \textit{VI} åbnes skal man trykke \textit{i} for at indsætte de tegn der er behov for og på en ny linje nederst, i file skrives:
\begin{center}
  alias cddesktop='cd ~/Desktop'
\end{center}
For at gennem trykkes der på escape knappe (esc), og så følge \textit{:wq}. \textit{:wq} er en vi kommando, w = write, q = quit. Når man har gemt, skal man genilæse sing \textit{.bashrc} fil, dette gøres således:
\begin{lstlisting}
	h271:intro_til_terminal tools$ .  ~/.bashrc
\end{lstlisting}

\section{CURL}
I dette eksemple benyttes material fra \href{http:www.ruby-lang.org}{www.ruby-lang.org} \par
CURL er en måde hvor på en fil kan hentes fra en server, det gøre ret simpelt med \textit{curl \#adresse til filen på serveren\#}, men outputtet giver ikke altid mening. Derfor kan man gøre brug af flaget -O, som gør at filen kan gives et navn. kommandostrukturen ser således \textit{curl \#navn\_på\_fil\# \#adresse til filen på serveren\#} og man skal være i den mappe, man ønsker filen skal ligge, eller give stien ved fil navnet. Se eksempel neden for 

\begin{lstlisting}
	h271h271:curl tools$ ls
	h271:curl tools$ curl -o ruby-2.1.0.tar.gz http://cache.ruby-lang.org/pub/ruby/2.1/ruby-2.1.0.tar.gz
	    % Total    % Received % Xferd  Average Speed   Time    Time     Time  Current
	                                   Dload  Upload   Total   Spent    Left  Speed
          100 14.3M  100 14.3M    0     0  6368k      0  0:00:02  0:00:02 --:--:-- 8655k
	h271:curl tools$ ls
	  ruby-2.1.0.tar.gz
\end{lstlisting}


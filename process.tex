\chapter{Processer}
Som de flest ved hvis de kommer fra eksempeltvis Windows eller Mac OS X, kan man se hvilke processer der køre via en jobliste på *nix systemer kan dette også gøres via kommandolinje og man kan også dræbe de processer som er løbet løbsk.
\section{TOP hvad kører}
Top hviser de mest belastne processer der køre og lidt informaiton om forskellig ting. Kommandoen kaldes ved at skrive \textit{top} og følgende output bliver vist  
\begin{lstlisting}
	h271:intro_til_terminal tools$ top
Processes: 188 total, 3 running, 4 stuck, 181 sleeping, 832 threads    11:22:47
Load Avg: 1.55, 1.62, 1.67  CPU usage: 10.8% user, 14.47% sys, 75.43% idle
SharedLibs: 175M resident, 0B data, 35M linkedit.
MemRegions: 39816 total, 2761M resident, 130M private, 770M shared.
PhysMem: 5441M used (1216M wired), 2750M unused.
VM: 454G vsize, 1310M framework vsize, 0(0) swapins, 0(0) swapouts.
Networks: packets: 2348958/2325M in, 2145578/1783M out.
Disks: 2436784/21G read, 207259/6427M written.

PID   COMMAND      %CPU      TIME     #TH  #WQ  #PORT #MREG MEM    RPRVT  PURG
4362  mdworker     0.0       00:00.09 4    0    54    58    1976K  1084K  0B
4152  top          11.7      00:34.09 1/1  0    23    35    2244K  2020K  0B
4149  com.apple.au 0.0       00:00.03 2    1    48    53    1328K  648K   0B
4148  com.apple.au 0.0       00:00.01 2    1    28    42    1004K  444K   0B
4147- com.apple.qt 0.0       00:00.07 2    0    73    81    2956K  1452K  0B
4146  com.apple.We 0.0       00:03.40 9    0    243   557   44M    40M    72K
4129  com.apple.Co 0.0       00:00.01 2    1    30    41    956K   400K   0B
4122  com.apple.Pr 0.0       00:00.01 2    1    35    42    980K   412K   0B
4119  Preview      0.0       00:06.53 4    0    205   444   35M    38M    20M
3926  Twitter      0.0       00:03.23 8    1    223   744   59M    47M    196K
3920  com.apple.hi 0.0       00:00.01 2    0    31    38    916K   368K   0B
3919- com.apple.qt 0.0       00:00.08 2    0    72    81    2932K  1428K  0B
3918  com.apple.We 3.9       00:23.06 11   2    267   1249+ 127M+  110M+  692K
3836  ScopedBookma 0.0       00:00.07 2    1    40    43    1800K  1360K  0B
\end{lstlisting}
\section{KILL og PKILL}
\textit{kill} og \textit{pkill} bruges til at dræbe processer, men på to forskellig måder.
\subsection*{KILL}
Basere sig på en nummer som kaldes \textit{PID} som er en process id, og som kan ses i outputtet for top. Alle processer har et \textit{PID} og de er alle unik (det vil sige ikke to er ens). Det at der er unikke tillader at man kan bruge \textit{kill} til at dræbe en process, lad os dræb Twitter processen som har PDI: 3926 
\begin{lstlisting}
	h271:intro_til_terminal tools$ kill 3926
\end{lstlisting}

\subsection*{PKILL - Vi dræber på navn}
Man kan også dræbe en process via dens kommando navn, så hvis man lige som under \textit{kill} vil dræbe Twitter kan det gøres således med \textit{pkill}
\begin{lstlisting}
	h271:intro_til_terminal tools$ pkill twitter
\end{lstlisting}
Med \textit{pkill} skal man være lidt mere påpasselig, da der godt kan køre flere processer med samme kommando navn. Der for anbefales det at man som nybegynder benytter \textit{kill}.


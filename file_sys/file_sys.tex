\chapter{Basal filesystem operationer}
En vigtig del af at benytte, en hver form for computer system, være det sig server eller desktop systemer.
Derfor vil de forskellige basale kommandoer, benyttet i sammenhæng med filsystemet beskrevet i dette kapitel.
\section{Print working directory}
\begin{lstlisting}
  pwd
\end{lstlisting}
Print working directory, printer den nu værende sti, som kommandoer vil blive udført i, uden andre sti parameter.
Hvis man eksempeltvis, køre kommandoen i sin hjemme mappe, ville retur værdien fra \emph{pwd}, være \emph{/Users/\#username\#}

\section{List files}
\begin{lstlisting}
  ls
\end{lstlisting}
List files tillader at list alle filer i en sti. Hvis der ikke gives en sti som input parameter, tages der udgangs punkt
i nuværende arbejdes sti, og alle ikke skjulte filer bliver printet. Hvis der også skal printes skjulte filer, kaldes
\emph{ls} med flaget \emph{-a}, hvor \emph{-a} står for all. Eksempler på kommando kald
\begin{lstlisting}
  ls
  ls -a
  ls /usr
  ls -a /usr
\end{lstlisting}

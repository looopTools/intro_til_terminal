\chapter{Introduktion}
Siden den personlige computeres oprindelse har der været to primære input enheder til computere. Den ene er pegeværktøj (mus, pointwheel eller trackball), det andet er tastaturet. 
I dag benytter de flest computerbruger en kombination af de to, til at naviger og benyttet deres computersystem. Men der findes også en brugergruppe som primært fortrækker at 
benytte sig af tastaturet. Folk der fortrækker tastaturet, har en værktøj de kan benyttet som kaldes Terminalen, til at udføre opgaver som ellers ofte udføres med musen. Terminalen er en applikation, som høre til kategorien kommandolinjeværktøjer (her efter CLI). \par
Terminalen tillader en bruger at kalde små til lidt større applikationer og systemer kommandoer. Og derved lader en bruger udføre opgaver som ellers kræver en form for pegeværktøj, så som flytning af filer. Terminalen applikationen findes på de flest UNIX og UNIX like systemer så som OS X, BSD og Linux. Bogen fokuser primært på funktionalitet som findes på de tre førnævnte operativ systemer.  
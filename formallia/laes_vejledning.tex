\chapter*{Læsevejledning}
Bogen er opdelt, således at kommandoer repræsenters og beskrives
i grupper, under et bestemt emne eksempeltvis \emph{Basal filesystem operationer}. \par
\textbf{Gloser}\\
I bogen forfindes, forskellig engelsk og eventuelle ukendte dansk udtryk,
disse findes beskrevet i appendiks \ref{app:gloser} \par
\textbf{Symboler}\\
I bogen benyttes symboler, som bruges i sammenhæng med terminalen. Disse symbolers, navn forfindes i appendiks \ref{app:symboler}\par
\textbf{Kommando notation}\\
Kommandoer er alle angivet med deres engelsk navn. Hver kommando er
repræsenter i starten af dennes afsnit således:
\begin{lstlisting}
  kommandonavn
\end{lstlisting}
Her efter kan kommandoer repræsenteres med flag, inputs og outputs (beskrives sener) så ledes:
\begin{lstlisting}
  kommandonavn flag input output
\end{lstlisting}

\section*{Til læseren}
Denne bog er henvendt til nybegynder og personer med meget lidt kendskab til
Terminal og BASH kommandoer gennrelt. Mere anvanceret bruger, kan måske
bruge denne bog som opslagsværk.

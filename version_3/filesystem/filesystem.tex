\chapter{Filsystem}
En del af det at benytte en computer, er at kunne navigere filsystemet hurtigt og effektivt. 
Derfor er det vigtigt at kunne benyttet de korrekte kommandoer, til disse arbejdes processer. 
Disse kommandoer vil blive dækket i dette kapitel.
\section{PWD}
\emph{P}rint \emph{W}orking \emph{D}irectory printer den sti, som man befinder sig i. Kommandoen tillader således en bruger at orientere sig om, hvor i filsystemet denne befinder sig.
\begin{lstlisting}[title=pwd eksempel]
	$:version_3 tools$ pwd
	/Users/tools/Documents/intro_til_terminal/version_3
\end{lstlisting}
\section{CD}
\emph{C}hange \emph{D}irectory ændre den sti, man befinder sig. Kommandoen tager et parameter, som skal være en sti eller \emph{..} som føre en bruger en niveau op i filsystemet. 
\begin{lstlisting}[title=cd eksempel]
	$:version_3 tools$ pwd
	/Users/tools/Documents/intro_til_terminal/version_3
	$:version_3 tools$ cd code/
	$:code tools$ pwd
	/Users/tools/Documents/intro_til_terminal/version_3/code
\end{lstlisting}
\section{LS}
\emph{L}ist directory content\emph{S} lister, som standard alle \textit{ikke skjulte} filer.
\begin{lstlisting}[title=ls eksempel]
	$:version_3 tools$ ls
	code		filesystem	main_book.log	
	compile.sh	introduktion	main_book.pdf	
	compile.sh~	main_book.aux	main_book.tex	
\end{lstlisting}
Hvis en bruger ønsker at få vist skjulte filer, tilføjer man flaget \textit{-a}, som betyder at 
man får vist alt
\begin{lstlisting}[title=ls -a eksempel]
	$:version_3 tools$ ls
	.	code		filesystem	main_book.log	
	..	compile.sh	introduktion	main_book.pdf	
\end{lstlisting}
\section{MKDIR}
\emph{M}a\emph{K}e \emph{DIR}ectory tillader en bruger at lave en mappe. Kommandoen tager et argument, som er navnet på den nye mappe.
\begin{lstlisting}[title=mkdir eksempel]
	$:demo tools$ ls
	$:demo tools$ mkdir dir_ex
	$:demo tools$ ls
	dir_ex
\end{lstlisting}
Hvis man ønsker at lave en mappe et andet sted i systemet, kan man give en fuld sti som parameter til kommandoen
\begin{lstlisting}[title=mkdir med sti eksempel]
	$:Desktop tools$ pwd
	/Users/tools/Desktop
	$:Desktop tools$ mkdir /Users/tools/Documents/intro_til_terminal/version_3/demo/dir_2_ex/
	$:Desktop tools$ ls /Users/tools/Documents/intro_til_terminal/version_3/demo/
	dir_2_ex	dir_ex
\end{lstlisting}
Dir 1: \textit{/Users/tools/Documents/intro\_til\_terminal/version\_3/demo/dir\_2\_ex/} \\
Dir 2: \textit{/Users/tools/Documents/intro\_til\_terminal/version\_3/demo/} \par
Det er også muligt at oprette flere mapper på engang, ved at skrive navne på mappe, med et mellemrum 
\begin{lstlisting}[title=mkdir med flere mapper eksempel]
	$:demo tools$ ls
	dir_ex
	$:demo tools$ mkdir dir_ex_2 dir_ex_3
	$:demo tools$ ls
	dir_ex		dir_ex_2	dir_ex_3
\end{lstlisting}


\section{RMDIR}
\emph{R}e\emph{M}ove \emph{DIR}ectory tillader at en bruger at fjerne en tom mappe. Kommandoen tager et argument, som er navnet på den nye mappe og kan som \textit{mkdir} også tage stien til en mappe.
\begin{lstlisting}[title=rmdir eksempel]	
	$:demo tools$ ls
	dir_2_ex	dir_ex
	$:demo tools$ rmdir dir_2_ex/
	$:demo tools$ ls
	dir_ex
\end{lstlisting}
\section{RM}
\emph{R}e\emph{M}ove tillader en at fjern en fil fra en mappe og tager pågældende fil som argument
\begin{lstlisting}[title=rmdir eksempel]	
	$:demo tools$ ls
	demo.txt	dir_ex
	$:demo tools$ rm dir_ex/
	rm: dir_ex/: is a directory
	$:demo tools$ rm demo.txt 
	$:demo tools$ ls
	dir_ex
\end{lstlisting}

Man kan også fjerne flere filer og under mapper og en hoved mappe på en gang. Dette gøres ved at bruge flaget \textit{-r}, hvor er betyder rekursivt.
\begin{lstlisting}[title=rmdir -r eksempel]	
	$:demo tools$ ls 
	demo_2	dir_ex
	$:demo tools$ ls demo_2/
	demo.txt
	$:demo tools$ rm demo_2/
	rm: demo_2/: is a directory
	$:demo tools$ rm -r demo_2/
	$:demo tools$ ls
	dir_ex
\end{lstlisting}

\section{MV}


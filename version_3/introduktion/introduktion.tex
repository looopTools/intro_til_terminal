\chapter{Introduktion}
I dette kapitel introduceres principperne omkring Terminal og hvordan den benyttes. \par
Terminalen er som før nævn en applikation, som giver en standard kommandolinje miljø. Hvis man installere en linux distribution som ArchLinux, får man adgang til et ren CLI-brugergrænseflade. Der imod installer man en linux distribution som Fedora som kommer med en grafiskbrugergrænseflade. Har man adgang til et program som emulere terminale, dette
program kan havde forskellige navne (eks. XTerm, Terminal eller Konsole). Begge typer kaldes terminal og som giver adgang til en masse applikationer og system programmer/procedure.
\section{Kommandoer}
For at kunne åbne programmer eller procedure i terminale, skal man kende den korrekt kommando. Det vil sige det "navn" som er associeret med en program eller en procedure. 
Eksempler på kommandoer kunne være:
\begin{itemize}
	\item ls
	\item mkdir
	\item df
\end{itemize}
\subsection*{Kommando struktur}
Alle kommandoer har en fælles struktur, der er får som ikke har den fælles struktur.
Denne struktur er derfor virkelig vigtigt at få slået fast og havde kendskab til. Denne struktur er således op bygget; kommando flag argumenter.\par
\textbf{Kommando} er som førnævnt navnet på det program eller den procedure man ønsker at kalde. \par
\textbf{Flag} er \textit{indstillinger} som fortæller et program eller en procedure hvordan en opgave skal udføres. En kommando kan tage op til flere flag.\par
\textbf{Argumenter} er det/de elementer som en kommando skal udføres på.
\begin{lstlisting}[title=kommando eksempel]
	$:ls -a ~/Desktop 
\end{lstlisting}


